\documentclass{article}

\usepackage[utf8]{inputenc}
\usepackage{mathtools, amssymb}
\usepackage{subcaption}
\usepackage{graphicx}
\usepackage{booktabs}

\graphicspath{ {./images/} }

\DeclarePairedDelimiter{\abs}{\lvert}{\rvert}
\DeclarePairedDelimiter{\norm}{\lVert}{\rVert}

\title{Image Point Motion}
\author{alex.kreimer}
\date{May 2016}

\begin{document}

\maketitle

\section{Image Point Motion}

Consider a 3-D point $\mathbf{X}=(X,Y,Z)^T$ and 2 cameras $P = K[I\ \mathbf{0}]$ and $P' = K[R\ \mathbf{t}]$.  The projections of the 3-D point onto the image planes are:
\begin{align}
\mathbf{x} &= P[\mathbf{X}^T1 ]^T = K\mathbf{X}.\\
\mathbf{x}' &= P'[\mathbf{X}^T1 ]^T = K[R\ \mathbf{t}][\mathbf{X}^T1 ]^T\label{eq:p2}.
\end{align}
$K$ is invertible, thus
\begin{equation}\label{eq:pinv}
\mathbf{X} = K^{-1}\mathbf{x}.
\end{equation}
Substitute~\ref{eq:pinv} into the~\ref{eq:p2} and simplify:
\begin{align}
\mathbf{x}' &=  K[R\ \mathbf{t}][(K^{-1}\mathbf{x})^T 1]^T\\
  &= [KR\ K\mathbf{t}][(K^{-1}\mathbf{x})^T 1]^T\\
  &= KRK^{-1}\mathbf{x} + K\mathbf{t}\label{eq:p3}.
\end{align}
Consider the case of pure translation: ($R=I$) Eq.~\ref{eq:p3}
becomes:
\begin{equation}
\mathbf{x}' = \mathbf{x} + K\mathbf{t}\label{eq:p4}.
\end{equation}
\textbf{Note:} $\mathbf{x},\mathbf{x}'$ are projective quantities (not the pixel coordinates). Converting them into the pixel coordinates yields (assuming $K = diag(f,f,1), \mathbf{t}=(t_1,t_2,t_3)^T$):
\begin{align}
(x,y)^T &= (fX/Z,fY/Z)^T.\\
(x',y')^T &= (f(X+t_1)/(Z+t_3),f(Y+t_2)/(Z+t_3)).
\end{align}
The equation (9.6) in the book states:
\begin{equation}
\mathbf{x}' = \mathbf{x} + K\mathbf{t}/Z
\end{equation}
In my notation it becomes:
\begin{align}
\mathbf{x}' &= (x,y,1)^T + K\mathbf{t}/Z\\
Z\mathbf{x}' &= (fX,fY,Z)^T + K\mathbf{t}
\end{align}
\end{document}

%%% Local Variables:
%%% mode: pdf
%%% TeX-master: t
%%% End:
